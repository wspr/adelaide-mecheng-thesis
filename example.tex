\documentclass{adelaide-mecheng-thesis}

\usepackage{amsmath,amssymb,bm,cool}% for maths
\usepackage{siunitx}% for typesetting units and numbers with units

% bibliography setup:
\usepackage[firstinits=true,backend=bibtex]{biblatex}% use "[style=authoryear]" alternatively.
\bibliography{papers}

\thesisdetails{
  author = {Zebb Prime and Will Robertson},
  title = {Example of a Mech.\ Eng.\ Thesis},
  submission-date = {November 14, 2012},
  honours-prelim, % "honours-final" for the final report
  sppa=true,
}

\graphicspath{{graphics/}}

\begin{document}


\maketitle

\frontmatter
\tableofcontents
\listoffigures
\listoftables

\mainmatter
\chapter{Introduction}

This is a class for getting start with writing a thesis or report in the School of Mechanical Engineering at The University of Adelaide.
This class is bases on \texttt{memoir}, for which you may refer for further information on customising the layout.
See one of the user guides, such as \texttt{lshort} for more information on learning \LaTeX.

\section{Maths}

An inline equation is short and cannot be referenced $a=b+c$. Note when you refer to numbers you want to use the \texttt{siunitx} package. It simplies things greatly so I can easily write $\ddot x=\SI{1.2e-2}{m/s^2}$, for example.

Maths in display form should \emph{always} be numbered and should be punctuated as if it were part of a sentence as in
\begin{equation}
F(s) = \mathcal{L}\{f(t)\} = \Int{f(t)\Exp{-st}}{t,-\infty,\infty} .
\eqlabel{laplace}
\end{equation}
Note the use of the \texttt{cool} package to help with inputting certain maths constructs. (The equation above is \eqref{laplace}.)

\section{Figures and tables}

Here is a reference to a graph; see \figref{graph}. And how about another one, \figref{subfig}, that has two subfigures \figref{subfiga,subfigb}?

\begin{figure}
\rule{4em}{4ex}
\caption{A figure that's actually just a rectangle.}
\figlabel{graph}
\end{figure}

\begin{figure}
\subfloat[Subfig one.\figlabel{subfiga}]{\rule{15em}{3ex}}
\hfil % equal space between and either side of the figures
      % or use \hfill to spread the figures to the margins
\subfloat[Subfig two.\figlabel{subfigb}]{\rule{10em}{2ex}}
\caption{A figure with two subfigures.}
\figlabel{subfig}
\end{figure}

\section{Tables}

See the example in \tabref{mytable}. Avoid vertical lines.
Traditionally, captions for tables come \emph{above} whereas captions for figures are below. (Consider a table that stretches over a page --- for which you'll need a package such as \texttt{longtable} --- for why this might be.)

\begin{table}
\caption{This is a tabular using the \texttt{booktabs} package.}
\tablabel{mytable}
\begin{tabular}{@{}lcc@{}}
\toprule
& \multicolumn{2}{@{}c@{}}{Variable} \\
\cmidrule{2-3}
& Symbol & Value \\
\midrule
Mass & $m$ & \SI{1.2e2}{kg} \\
Acceleration & $\ddot x$ & \SI{0.4}{m/s^2} \\
\bottomrule
\end{tabular}
\end{table}

\section{Literature review}

You can reference papers either inline such as when I want to talk about the work of \textcite{fahey1998-sportseng}. Or they can be parenthetical \parencite{hubbard1984-biomech}. I dislike using parethetical references as sentence objects, such as writing `see \parencite{foss2007-sv} for further context'. Write at least `\dots Ref.\ \parencite{foss2007-sv}\dots ' or ideally reword the sentence.

The package used for referencing is \texttt{biblatex}. It has many customisation features and can also be used to reference in an author--year style. See its manual for more information.

\printbibliography

\appendix
\chapter{An appendix}



\end{document}